\documentclass{book}
\usepackage{xepersian}
\settextfont{Titr}
\author{اکبر اکبری}
\title{چرا اکبر؟}
\pagestyle{headings}
\begin{document}
	\maketitle
 	\tableofcontents
 	%\pagestyle{headings}
 	\chapter{پیشگفتار}
 	\label{key}
 	این کلمه ی فصل رو خودش نوشته! اینجا خیلی خبری نیست. \footnote{یه خبر کوچیک اینجا در پانویس!}
 	\chapter{فرع کار}
 	\section{مقدمه}
	\section{بخش اول}
	\subsection{قسمت اول}
	\subsubsection[تو فهرست اینو نشون بده نه اصلی رو]{پارهٔ اول}
	\paragraph{قضیه}
	اگر کسی اکبر باشد آنگاه اکبر است.
	\paragraph{مثال}
	این هم مثال
	
	\paragraph{پاراگراف دوم}
	ویلارد ون ارمن کواین سییییییبیییییل ندارد ویلارد ون ارمن کواین سییییییبیییییل نداردویلارد ون ارمن کواین سییییییبیییییل ندارد ویلارد ون ارمن کواین سییییییبیییییل نداردویلارد ون ارمن کواین سییییییبیییییل ندارد ویلارد ون ارمن کواین سییییییبیییییل نداردویلارد ون ارمن کواین سییییییبیییییل ندارد ویلارد ون ارمن کواین سییییییبیییییل نداردویلارد ون ارمن کواین سییییییبیییییل ندارد ویلارد ون ارمن کواین سییییییبیییییل نداردویلارد ون ارمن کواین سییییییبیییییل ندارد
	\subsubsection*{پارهٔ دوم}
	 ویلارد ون ارمن کواین سییییییبیییییل نداردویلارد ون ارمن کواین سییییییبیییییل ندارد ویلارد ون ارمن کواین سییییییبیییییل نداردویلارد ون ارمن کواین سییییییبیییییل ندارد ویلارد 
	\subsection{قسمت دوم}
	ون ارمن کواین سییییییبیییییل نداردویلارد ون ارمن کواین سییییییبیییییل ندارد ویلارد ون ارمن کواین سییییییبیییییل نداردویلارد ون ارمن کواین سییییییبیییییل ندارد ویلارد ون ارمن کواین سییییییبیییییل نداردویلارد ون ارمن کواین سییییییبیییییل ندارد ویلارد ون ارمن کواین سییییییبیییییل نداردویلارد ون ارمن کواین سییییییبیییییل ندارد ویلارد ون ارمن کواین سییییییبیییییل نداردویلارد ون ارمن کواین سییییییبیییییل ندارد ویلارد ون ارمن کواین سییییییبیییییل نداردویلارد ون ارمن کواین سییییییبیییییل ندارد ویلارد ون ارمن کواین سییییییبیییییل ندارد
	ویلارد ون ارمن کواین سییییییبیییییل ندارد ویلارد ون ارمن کواین سییییییبیییییل نداردویلارد ون ارمن کواین سییییییبیییییل ندارد ویلارد ون ارمن کواین سییییییبیییییل نداردویلارد ون ارمن کواین سییییییبیییییل ندارد ویلارد ون ارمن کواین سییییییبیییییل نداردویلارد ون ارمن کواین سییییییبیییییل ندارد ویلارد ون ارمن کواین سییییییبیییییل نداردویلارد ون ارمن کواین سییییییبیییییل ندارد ویلارد ون ارمن کواین سییییییبیییییل نداردویلارد ون ارمن کواین سییییییبیییییل ندارد ویلارد ون ارمن کواین سییییییبیییییل نداردویلارد ون ارمن کواین سییییییبیییییل ندارد ویلارد ون ارمن کواین سییییییبیییییل نداردویلارد ون ارمن کواین سییییییبیییییل ندارد ویلارد ون ارمن کواین سییییییبیییییل نداردویلارد ون ارمن کواین سییییییبیییییل ندارد ویلارد ون ارمن کواین سییییییبیییییل نداردویلارد ون ارمن کواین سییییییبیییییل ندارد ویلارد ون ارمن کواین سییییییبیییییل نداردویلارد ون ارمن کواین سییییییبیییییل ندارد ویلارد ون ارمن کواین سییییییبیییییل نداردویلارد ون ارمن کواین سییییییبیییییل ندارد ویلارد ون ارمن کواین سییییییبیییییل نداردویلارد ون ارمن کواین سییییییبیییییل ندارد ویلارد ون ارمن کواین سییییییبیییییل نداردویلارد ون ارمن کواین سییییییبیییییل ندارد ویلارد ون ارمن کواین سییییییبیییییل ندارد ویلارد ون ارمن کواین سییییییبیییییل ندارد ویلارد ون ارمن کواین سییییییبیییییل نداردویلارد ون ارمن کواین سییییییبیییییل ندارد ویلارد ون ارمن کواین سییییییبیییییل نداردویلارد ون ارمن کواین سییییییبیییییل ندارد ویلارد ون ارمن کواین سییییییبیییییل نداردویلارد ون ارمن کواین سییییییبیییییل ندارد ویلارد ون ارمن کواین سییییییبیییییل نداردویلارد ون ارمن کواین سییییییبیییییل ندارد \mbox{این کلمه ها با هم تو یه خط-} \mbox{این کلمه ها با هم تو یه خط-} \mbox{این کلمه ها با هم تو یه خط-} \mbox{این کلمه ها با هم تو یه خط-} \mbox{این کلمه ها با هم تو یه خط-} \mbox{این کلمه ها با هم تو یه خط-} \mbox{این کلمه ها با هم تو یه خط-} \mbox{این کلمه ها با هم تو یه خط-} \mbox{این کلمه ها با هم تو یه خط-} \mbox{این کلمه ها با هم تو یه خط-}
	
	\section{بخش دوم}
	اکبر را از نزدیک ملاقات می کنیم. 
	
	همه شب من اختر شمرم
	\\
	به این رابطه ی اکبری دقت کنید:
	\\
	اینجا پاراگراف نشکست
	
	اینجا شکست. صفحه ی قدیم
	\newpage
	یووو صفحه ی جدید
	\begin{equation}
		e = m \cdot c^2 \; ,
	\end{equation}
	
	\newpage
	صفحه ی جدیدتر
	\\
	
	
	\fbox{مولانا فرمود:}
	٬٬‌جان نباشد جز خبر در آزمون، هر که را افزون خبر جانش فزون؟؟
	\begin{quote}
		این را هم فرموده بود آن پیر
	\end{quote}
	
	 ممد
	 \underline{مشکاتیان}
	
	$x^۲ = {{xy}^2}^4$
	
	۱۲۳ - ۴۵
	\\
	\\
	
		می شه درسته از مقاله ریخت اینجا
		\\

	\begin{flushleft}
		\lr{
		If $A$ is an $n \times n$ Hermitian matrix with eigenvalues $\lambda_1(A),\dots,\lambda_n(A)$ and $i,j = 1,\dots,n$, then the $j^{\mathrm{th}}$ component $v_{i,j}$ of a unit eigenvector $v_i$ associated to the eigenvalue $\lambda_i(A)$ is related to the eigenvalues $\lambda_1(M_j),\dots,\lambda_{n-1}(M_j)$ of the minor $M_j$ of $A$ formed by removing the $j^{\mathrm{th}}$ row and column by the formula
		$$ |v_{i,j}|^2\prod_{k=1;k\neq i}^{n}\left(\lambda_i(A)-\lambda_k(A)\right)=\prod_{k=1}^{n-1}\left(\lambda_i(A)-\lambda_k(M_j)\right)\,.$$
		We refer to this identity as the \emph{eigenvector-eigenvalue identity}.  Despite the simple nature of this identity and the extremely mature state of development of linear algebra, this identity was not widely known until very recently.  In this survey we describe the many times that this identity, or variants thereof, have been discovered and rediscovered in the literature (with the earliest precursor we know of appearing in 1834).  We also provide a number of proofs and generalizations of the identity.  
	}
	\end{flushleft}

ما را ببر به پیشگفتار \ref{key}
\\
\\
نمیبرد...میبرد؟


	
	
\end{document}